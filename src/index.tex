% !TeX root = index.tex
% define document class
\documentclass{scrreprt}

% set variables
\newcommand{\varAuthor}{Isaac Lins}
\newcommand{\varCandidate}{\textbf{\varAuthor} \\ isaac.lins@swisscom.com} % Kandidat/in
\newcommand{\varResponsibleSpecialist}{\textbf{ Dimosthenis Georgokitsos } \\ dimosthenis.georgokitsos@swisscom.com } % verantwortliche Fachkraft
\newcommand{\varVocationalTrainer}{\textbf{Thomas Albori} \\ thomas.albori@swisscom.com} % Berufsbildner/in
\newcommand{\varPrimaryExpert}{\textbf{Gabriel Lucaci}  \\ info@mail.lucaci.ch} % Hauptexperte
\newcommand{\varSecondaryExpert}{\textbf{Michael Stolz} \\ michael.stolz@gmail.com} % Nebenexperte
\newcommand{\varCompany}{Swisscom AG} % Firmenname
\newcommand{\varCompanyDepartment}{TDR} % Abteilungsname
\newcommand{\varTitle}{Schichtplanungs-CLI für TDR} % Titel der Arbeit
\newcommand{\varVersion}{0.1} % Versionsnummer: Pro IPA-Tag um 0.1 erhöhen
\newcommand{\varExaminationBoard}{Prüfungskomission 19} % Prüfungsorganisation
\newcommand{\varExaminationBoardDepartment}{Informatik Applikationsentwicklung} % Fachrichtung

% apply ipa package
\usepackage{../lib/ipa}
\usepackage{../lib/tikz-uml}

\lohead[\varTitle]{\varTitle}
\lofoot[\today]{\today}
% see B6.5
\cfoot[\varAuthor\ / \varCompany\\Version \varVersion]{\varAuthor\ / \varCompany\\Version \varVersion}
\rofoot[Seite \pagemark{} von \pageref{LastPage}]{Seite \pagemark{} von \pageref{LastPage}}

% load sources
\addbibresource{sources.bib}

% make glossaries
\makenoidxglossaries

% define glossary entries
\input{glossaries}

% create document
\begin{document}

  % set page numbering
  \pagenumbering{roman}

  % set page style
  \pagestyle{scrheadings}

  % include title page
  \thispagestyle{empty}
  
  \input{pages/title}

  \newpage
  \TileWallPaper{\paperwidth}{\paperheight}{images/background.pdf}

  % see A1.3
  % see B6.3
  % generate the table of contents
  \tableofcontents

  % finish page
  \clearpage

  % use the arabic numbering system
  \pagenumbering{arabic}

  % reset page counter
  \setcounter{page}{1}

  % see B6.1a
  % create a phantom toc entry for "Umfeld und Ablauf"
  \clearpage\phantomsection\addcontentsline{toc}{part}{Umfeld und Ablauf}

  \input{chapters/task}
  \input{chapters/declaration}
  \input{chapters/organisation}
  \input{chapters/timeplan}
  \input{chapters/journal}

  % see B6.1a
  % create a phantom toc entry for "Projekt"
  \clearpage\phantomsection\addcontentsline{toc}{part}{Projekt}

  \input{chapters/summary}
  \input{chapters/inform}
  \input{chapters/plan}
  \input{chapters/decide}
  \input{chapters/implement}
  \input{chapters/check}
  \input{chapters/evaluate}

  % see B6.6
  % create a phantom toc entry for the index/glossary table
  \clearpage\phantomsection\addcontentsline{toc}{part}{Glossar}

  % generate glossary
  \printnoidxglossary[title={Glossar}]

  % create a phantom toc entry for the figures table
  \clearpage\phantomsection\addcontentsline{toc}{part}{Abbildungsverzeichnis}

  % generate figures table
  \listoffigures

  % create a phantom toc entry for the literature table
  \clearpage\phantomsection\addcontentsline{toc}{part}{Literaturverzeichnis}

  % generate bibliography
  \printbibliography[title=Literaturverzeichnis]

  % defines the beginning of the appendix
  \appendix

  % create a phantom toc entry for "Projekt"
  \clearpage\phantomsection\addcontentsline{toc}{part}{Anhang}

  \input{appendix/source}

\end{document}
